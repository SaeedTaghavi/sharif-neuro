
\section{Introduction}

%\subsection{The Model and the Problem}
\subsection{The Problem}

\begin{frame}{Neural synchrony in brain disorders}


Certain brain disorders are associated with abnormaly neural synchronization.
%  Dysfunctional communication resulting from an inability to properly modulate oscillatory activity, either through hypo- or hyper-synchrony, has been implicated in a number of neurological disorders.
%In Parkinson's disease (PD), movement impairment is correlated with exaggerated beta frequency oscillations in the cerebral cortex and subthalamic numcleus(STN).
%There is evidence for enhanced synchronized $\beta$-band activity prior to movement preparation and during visuo-motor coordination.
\includegraphics[width=\linewidth]{EToscil}
%  \begin{block}{remember}
%      Shanon entropy\\
%
%      \begin{equation*}
%        H=\Sigma_{i=1}^{N} -p_{i}\log(p_{i})
%      \end{equation*}
%  \end{block}
\end{frame} 

\begin{frame}[t]{General formalization of haplotyping.}
  \begin{block}{Inputs}
    \begin{itemize}
    \item A \alert{genotype matrix} $G$.
    \item The \alert{rows} of the matrix are \alert{taxa / individuals}.
    \item The \alert{columns} of the matrix are \alert{SNP sites /
        characters}. 
    \end{itemize}
  \end{block}
  \begin{block}{Outputs}
    \begin{itemize}
    \item A \alert{haplotype matrix} $H$.
    \item Pairs of rows in $H$ \alert{explain} the rows of $G$.
    \item The haplotypes in $H$ are \alert{biologically plausible}. 
    \end{itemize}
  \end{block}
\end{frame}


\begin{frame}[t]{Our formalization of haplotyping.}
  \begin{block}{Inputs}
    \begin{itemize}
    \item A genotype matrix $G$.
    \item The rows of the matrix are individuals / taxa.
    \item The columns of the matrix are SNP sites / characters.
    \item<alert@1->
      The problem is directed: one haplotype is known.
    \item<alert@1->
      The input is biallelic: there are only two homozygous
      states (0 and 1) and one heterozygous state (2).
    \end{itemize}
  \end{block}
  \begin{block}{Outputs}
    \begin{itemize}
    \item A haplotype matrix $H$.
    \item Pairs of rows in $H$ explain the rows of $G$.
    \item<alert@1> The haplotypes in $H$ form a perfect phylogeny.
    \end{itemize}
  \end{block}
\end{frame}


\begin{frame}{We can do perfect phylogeny haplotyping efficiently, but
    \dots}
  \begin{enumerate}
  \item \alert{Data may be missing.}
    \begin{itemize}
    \item This makes the problem NP-complete \dots
    \item \dots even for very restricted cases.
    \end{itemize}
    \textcolor{green!50!black}{Solutions:}
    \begin{itemize}
    \item Additional assumption like the rich data hypothesis. 
    \end{itemize}
  \item \alert{No perfect phylogeny is possible.}
    \begin{itemize}
    \item This can be caused by chromosomal crossing-over effects.
    \item This can be caused by incorrect data.
    \item This can be caused by multiple mutations at the same sites.
    \end{itemize}
    \textcolor{green!50!black}{Solutions:}
    \begin{itemize}
    \item Look for phylogenetic networks.
    \item Correct data.
    \item<alert@1->
       Find blocks where a perfect phylogeny is possible.
    \end{itemize}
  \end{enumerate}
\end{frame}


\subsection{The Integrated Approach}

\begin{frame}{How blocks help in perfect phylogeny haplotyping.}
  \begin{enumerate}
  \item Partition the site set into overlapping contiguous blocks.
  \item Compute a perfect phylogeny for each block and combine them.
  \item Use dynamic programming for finding the partition.
  \end{enumerate}

  \begin{tikzpicture}
    \useasboundingbox (0,-1) rectangle (10,2);
    
    \draw[line width=2mm,dash pattern=on 1mm off 1mm]
      (0,1) -- (9.99,1) node[midway,above] {Genotype matrix}
      (0,0.6666) -- (9.99,0.6666)
      (0,0.3333) -- (9.99,0.3333)
      (0,0) -- (9.99,0) node[midway,below] {\only<1>{no perfect phylogeny}};

    \begin{scope}[xshift=-.5mm]
      \only<2->
      {
        \draw[red,block]            (0,.5)   -- (3,.5)
          node[midway,below] {perfect phylogeny};
      }
        
      \only<3->
      {
        \draw[green!50!black,block] (2.5,.5)   -- (7,.5)
          node[pos=0.6,below] {perfect phylogeny};
      }

      \only<4->
      {
        \draw[blue,block]           (6.5,.5) -- (10,.5)
          node[pos=0.6,below] {perfect phylogeny};
      }
    \end{scope}
  \end{tikzpicture}
\end{frame}

\begin{frame}{Objective of the integrated approach.}
  \begin{enumerate}
  \item Partition the site set into \alert{noncontiguous} blocks. 
  \item Compute a perfect phylogeny for each block and combine them. 
  \item<alert@1-> Compute partition while computing perfect
    phylogenies. 
  \end{enumerate}

  \begin{tikzpicture}
    \useasboundingbox (0,-1) rectangle (10,2);

    \draw[line width=2mm,dash pattern=on 1mm off 1mm]
      (0,1) -- (9.99,1) node[midway,above] {Genotype matrix}
      (0,0.6666) -- (9.99,0.6666)
      (0,0.3333) -- (9.99,0.3333)
      (0,0) -- (9.99,0) node[midway,below] {\only<1>{no perfect phylogeny}};

    \only<2->
    {
      \begin{scope}[xshift=-0.5mm]
        \draw[red,block] (0,.5)   -- (3,.5) 
          node[midway,below] {perfect phylogeny}
                         (8,.5) -- (9,.5);

        \draw[green!50!black,block]
          (3,.5)   -- (6,.5)
            node[pos=0.6,below] {perfect phylogeny}
          (6.4,.5)   -- (8,.5)
          (9,.5) -- (10,.5);

        \draw[blue,block] (6,.5) -- (6.4,.5)
          node[midway,below=5mm] {perfect phylogeny};
      \end{scope}
    }
  \end{tikzpicture}
\end{frame}


\begin{frame}{The formal computational problem.}
  We are interested in the computational complexity of \\
  \alert{the function \alert{$\chi_{\operatorname{PP}}$}}:
  \begin{itemize}
  \item It gets genotype matrices as input.
  \item It maps them to a number $k$.
  \item This number is minimal such that the sites can be
    covered by $k$ sets, each admitting a perfect phylogeny.
    \\
    (We call this a \alert{pp-partition}.)
  \end{itemize}
\end{frame}


\subsection{Phase Response Curve}
\begin{frame}{The Phase Response Curve (PRC)}
  We are interested in the computational complexity of \\
  \alert{the function \alert{$\chi_{\operatorname{PP}}$}}:
  \begin{itemize}
  \item It gets genotype matrices as input.
  \item It maps them to a number $k$.
  \item This number is minimal such that the sites can be
    covered by $k$ sets, each admitting a perfect phylogeny.
    \\
    (We call this a \alert{pp-partition}.)
  \end{itemize}
\end{frame}