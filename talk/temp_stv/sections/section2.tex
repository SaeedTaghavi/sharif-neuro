\section{Results}

\subsection{Perfect Path Phylogenies}

\begin{frame}{Automatic optimal pp-partitioning is hopeless, but\dots}
  \begin{itemize}
  \item The hardness results are \alert{worst-case} results for\\
    \alert{highly artificial inputs}.
  \item \alert{Real biological data} might have special properties
    that make the problem \alert{tractable}.
  \item One such property is that perfect phylogenies are often
    perfect \alert{path} phylogenies:

    In HapMap data, in 70\% of the blocks where a perfect phylogeny
    is possible a perfect path phylogeny is also possible.
  \end{itemize}  
\end{frame}


\begin{frame}{Example of a perfect path phylogeny.}
  \begin{columns}[t]
    \column{.3\textwidth}
    \begin{exampleblock}{Genotype matrix}
      $G\colon$
      \begin{tabular}{ccc}
        A & B & C \\\hline
        2 & 2 & 2 \\
        0 & 2 & 0 \\
        2 & 0 & 0 \\
        0 & 2 & 2 
      \end{tabular}
    \end{exampleblock}

    \column{.3\textwidth}
    \begin{exampleblock}{Haplotype matrix}
      $H\colon$
      \begin{tabular}{ccc}
        A & B & C \\\hline
        1 & 0 & 0 \\
        0 & 1 & 1 \\
        0 & 0 & 0 \\
        0 & 1 & 0 \\
        0 & 0 & 0 \\
        1 & 0 & 0 \\
        0 & 0 & 0 \\
        0 & 1 & 1 
      \end{tabular}
    \end{exampleblock}

    \column{.4\textwidth}
    \begin{exampleblock}{Perfect path phylogeny}
      \begin{center}
        \begin{tikzpicture}[auto,thick]
          \tikzstyle{node}=%
          [%
            minimum size=10pt,%
            inner sep=0pt,%
            outer sep=0pt,%
            ball color=example text.fg,%
            circle%
          ]
        
          \node [node] {} [->]
            child {node [node] {} edge from parent node[swap]{A}}
            child {node [node] {}
              child {node [node] {} edge from parent node{C}}
              edge from parent node{B}
            };
        \end{tikzpicture}
      \end{center}
    \end{exampleblock}
  \end{columns}
\end{frame}


\begin{frame}{The modified formal computational problem.}
  We are interested in the computational complexity of \\
  the function $\chi_{{\operatorname{PPP}}}$:
  \begin{itemize}
  \item It gets genotype matrices as input.
  \item It maps them to a number $k$.
  \item This number is minimal such that the sites can be
    covered by $k$ sets, each admitting a perfect \alert{path} phylogeny.
    \\
    (We call this a ppp-partition.)
  \end{itemize}
\end{frame}



\subsection{Tractability of PPP-Partitioning of Genotype Matrices}

\begin{frame}{Good news about ppp-partitions of genotype matrices.}
  \begin{theorem}
    \alert{Optimal ppp-partitions of genotype matrices} can be
    computed in \alert{polynomial time}. 
  \end{theorem}
  \begin{block}{Algorithm}
    \begin{enumerate}
    \item Build the following partial order:
      \begin{itemize}
      \item Can one column be above the other in a phylogeny?
      \item Can the columns be the two children of the root of a
        perfect path phylogeny?
      \end{itemize}
    \item Cover the partial order with as few compatible chain pairs 
      as possible. 

      For this, a maximal matching in a special graph needs to be
      computed.
    \end{enumerate}
  \end{block}
  \hyperlink{algorithm<1>}{\beamergotobutton{The algorithm in action}}
  \hypertarget{return}{}
\end{frame}
