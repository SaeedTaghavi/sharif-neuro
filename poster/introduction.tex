\headerbox{Intruduction}{name=introduction,column=0, span=2}{
\large{
While synchronized oscillations within and between brain areas facilitate normal brain processing, excess synchrony usually is accompanied by a brain disease. A prominent example is the amplified persistent beta-frequency ($\sim$ 20 Hz) oscillations recorded from the cortex and subthalamic nucleus of Parkinsonian brains \cite{asllani2018minimally, holt2019phase}. Deep brain stimulation (DBS) is known to be an effective treatment for a variety of neurological disorders, including Parkinson's disease and essential tremor (ET). A common procedure is to impose a train of pulses with constant frequency via electrodes implanted into the brain. New 'closed-loop' approach involves delivering stimulation according to the ongoing brain activity and could improve in terms of efficiency and reduce side effects. The success of closed-loop DBS depends on the design of a stimulation strategy that minimizes oscillations in neural activity associated with symptoms \cite{weerasinghe2019predicting}. An important step to this end is to construct a mathematical model, which can describe how the brain oscillations should change when stimulation is applied at a particular state of the system.  
}
\vspace{0.4em}
}

\headerbox{DBS strategies}{name=introplot,column=2, span=1}{
\vspace*{-0.17cm}

\center{
\includegraphics[width=.75\linewidth]{fig/dbs_strategy.png} 
}

}